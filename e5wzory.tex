\documentclass{article}

\usepackage[T1]{fontenc}
\usepackage[polish]{babel}
\usepackage[utf8]{inputenc}
\usepackage{amsmath}
\usepackage{hyperref}
\usepackage{tablefootnote}
\usepackage{graphicx}
\usepackage{siunitx}
\usepackage{listings}
\usepackage{xcolor}

\lstset{frame=tb,
  language=Python,
  aboveskip=3mm,
  belowskip=3mm,
  showstringspaces=false,
  columns=flexible,
  basicstyle={\small\ttfamily},
  numbers=none,
  numberstyle=\tiny\color{gray},
  keywordstyle=\color{blue},
  commentstyle=\color{dkgreen},
  stringstyle=\color{green},
  breaklines=true,
  breakatwhitespace=true,
  tabsize=3
}

\graphicspath{ {./media/} }

\begin{document}


\section{Użyte wzory}

\subsection{Wyniki}
Przy pierwszych pomiarach (zmienianie prądu płynącego w ramce) chcąc wyznaczyć 
wartość indukcji pola magnetycznego B posłużymy się wzorem:
\begin{gather*}
	F = ILB \\
	(m-m_0)g = ILB \\
	m(I) = \frac{LB}{g}I + m_0
\end{gather*} 
gdzie $m_0$ - masa samej ramki,\\
$m$ - masa pozorna ramki przy płynącym przez nią prądzie I \\
$L=0,1$ m\footnote{\url{https://pg.edu.pl/files/ftims/2021-03/cwiczenieE5.pdf}} - długość odcinka przewodnika oddziałującego z polem magnetycznym. \\
Korzystając z metody najmniejszych kwadratów otrzymamy wspł.  kierunkowy prostej
\begin{gather*}
	a = \frac{LB}{g} \rightarrow B = \frac{ag}{L}
\end{gather*}
W następym pomiarze (zmiana prądu płynącego przez uzwojenie elektromagnesu) skorzystamy z zależności
\begin{gather*}
	B = \frac{(m-m_0)g}{IL}
\end{gather*}
przy określeniu zależności $B(I_m)$, gdzie $I_m$ - natężenie prądu w uzwojeniu elektromagnesu.

\subsection{Niepewności}
Niepewność indukcji pola magnetycznego wyznaczonego metodą najmniejszych kwadratów wyznaczymy z odpowiednich wzorów  \footnote{\url{https://ftims.pg.edu.pl/documents/10673/20436990/wstep.pdf}}\:

\begin{gather*}
		u_a = \sqrt{\frac{n}{n-2} * \frac{\Sigma y_i^2 - a\Sigma x_iy_i}{n\Sigma x_i^2}} 
\end{gather*}
skąd
\begin{gather*}
		u_B= |\frac{\partial B}{\partial a}| u_a = \frac{g}{L}u_a
\end{gather*}
W przypadku zależności $B(I_m)$ niepewność B wyznaczamy jako niepewność funkcji złożonej zmiennych 
$m, m_0, I$:

\begin{gather*}
		\Delta B = |\frac{\partial B}{\partial m_0}|\Delta m_0 + |\frac{\partial B}{\partial m}|\Delta m + |\frac{\partial B}{\partial I}|\Delta I = \frac{g}{IL}(\Delta m + \Delta m_0 + \frac{m-m_0}{I}\Delta I)
\end{gather*}
gdzie $\Delta m_0 =\Delta m = 0,01$ g oraz $\Delta I = 0,02$ A. (to jeszcze do sprawdzenia czy takie niepewności były)


\end{document}