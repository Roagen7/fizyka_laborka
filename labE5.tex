\documentclass{article}

\usepackage[T1]{fontenc}
\usepackage[polish]{babel}
\usepackage[utf8]{inputenc}
\usepackage{amsmath}
\usepackage{hyperref}
\usepackage{tablefootnote}
\usepackage{graphicx}
\usepackage{siunitx}
\usepackage{listings}
\usepackage{xcolor}

\lstset{frame=tb,
  language=Python,
  aboveskip=3mm,
  belowskip=3mm,
  showstringspaces=false,
  columns=flexible,
  basicstyle={\small\ttfamily},
  numbers=none,
  numberstyle=\tiny\color{gray},
  keywordstyle=\color{blue},
  commentstyle=\color{dkgreen},
  stringstyle=\color{green},
  breaklines=true,
  breakatwhitespace=true,
  tabsize=3
}

\graphicspath{ {./media/} }

\begin{document}

\section{Cel ćwiczenia}
Celem ćwiczenia jest pomiar siły elektrodynamicznej (przy pomocy wagi) działającej na odcinek przewodnika z prądem, który został umieszczony w jednorodnym polu magnetycznym. Badana jest zależność tej siły od natężenia prądu płynącego w przewodniku i od indukcji pola magnetycznego. Na podstawie przeprowadzonych pomiarów wyznaczana jest wartość indukcji pola magnetycznego

\section{Wyznaczanie siły działającej na fragment przewodnika}
\subsection{pomierzone dane}

$U$ = 12 V
\begin{center}
\begin{tabular}{ c | c | c}
I [A] & m [g] & F [N]\\
\hline
0,5 & 38,19  & 8.92\\
1,0 & 39,08  & 17.64\\
1,5 & 39,96  & 26.26\\
2,0 & 40,85  & 34.99\\
2,5 & 41,60  & 42.34\\
3,0 & 42,74  & 53.50\\
3,5 & 43,67  & 62.62\\
4,0 & 44,62  & 71.93\\
4,5 & 45,44  & 79.97\\
5,0 & 46,39  & 89.28\\
\end{tabular}
\end{center}

$U$ = 6 V
\begin{center}
\begin{tabular}{ c | c | c}
I [A] & m [g] & F [N]\\
\hline
0,5 & 37,61  & 3.23\\
1,0 & 38,01  & 7.15\\
1,5 & 38,38  & 10.78\\
2,0 & 38,77  & 14.60\\
2,5 & 39,14  & 18.23\\
3,0 & 39,52  & 21.95\\
3,5 & 39,98  & 26.46\\
4,0 & 40,31  & 29.69\\
4,5 & 40,74  & 33.90\\
5,0 & 41,11  & 37.53\\
\end{tabular}
\end{center}

$I$ - prąd płynądy przez ramkę \\
$m$ - masa pozorna ramki przy płynącym przez nią prądzie I \\\\
$m_0$ - masa ramki = 37,28 g\\
$g$ - przyspiespieszenie ziemskie $\approx$ 9,80 [$\frac{m}{s^2}$]

\section{Określenie zależności indukcji magnetycznej elektromagnesu}

\begin{center}
  \begin{tabular}{ c | c | c | c}
  U [V] & I [A] & m [g] & B [T]\\
  \hline
  2 & 0.04 & 37.67 & 955.50\\
  4 & 0.2 & 38.96 & 823.20\\
  6 & 0.36 & 40.27 & 813.94\\
  8 & 0.53 & 41.68 & 813.58\\
  10 & 0.7 & 43.11 & 816.20\\
  12 & 0.86 & 44.52 & 825.02\\
  \end{tabular}
\end{center}

$I$ - natężenie prądu w uzwojeniu elektromagnesu \\
$m$ - masa pozorna ramki przy płynącym przez nią prądzie I \\\\
$m_0$ - masa ramki = 37,28 g\\
$g$ - przyspiespieszenie ziemskie $\approx$ 9,80 [$\frac{m}{s^2}$]

\section{Wartość indukcji magnetycznej w szczelinie elektomagnesu}

a, b = 1.83 37.21
(wyszlo w 1 przykladzie a = 1.82 więc się zgadza)

\section{Użyte wzory}

\subsection{Wyniki}
Przy pierwszych pomiarach (zmienianie prądu płynącego w ramce) chcąc wyznaczyć 
wartość indukcji pola magnetycznego B posłużymy się wzorem:
\begin{gather*}
	F = ILB \\
	(m-m_0)g = ILB \\
	m(I) = \frac{LB}{g}I + m_0
\end{gather*} 
gdzie $m_0$ - masa samej ramki,\\
$m$ - masa pozorna ramki przy płynącym przez nią prądzie I \\
$L=0,1$ m\footnote{\url{https://pg.edu.pl/files/ftims/2021-03/cwiczenieE5.pdf}} - długość odcinka przewodnika oddziałującego z polem magnetycznym. \\
Korzystając z metody najmniejszych kwadratów otrzymamy wspł.  kierunkowy prostej
\begin{gather*}
	a = \frac{LB}{g} \rightarrow B = \frac{ag}{L}
\end{gather*}
W następym pomiarze (zmiana prądu płynącego przez uzwojenie elektromagnesu) skorzystamy z zależności
\begin{gather*}
	B = \frac{(m-m_0)g}{IL}
\end{gather*}
przy określeniu zależności $B(I_m)$, gdzie $I_m$ - natężenie prądu w uzwojeniu elektromagnesu.

\subsection{Niepewności}
Niepewność indukcji pola magnetycznego wyznaczonego metodą najmniejszych kwadratów wyznaczymy z odpowiednich wzorów  \footnote{\url{https://ftims.pg.edu.pl/documents/10673/20436990/wstep.pdf}}\:

\begin{gather*}
		u_a = \sqrt{\frac{n}{n-2} * \frac{\Sigma y_i^2 - a\Sigma x_iy_i}{n\Sigma x_i^2}} 
\end{gather*}
skąd
\begin{gather*}
		u_B= |\frac{\partial B}{\partial a}| u_a = \frac{g}{L}u_a
\end{gather*}
W przypadku zależności $B(I_m)$ niepewność B wyznaczamy jako niepewność funkcji złożonej zmiennych 
$m, m_0, I$:

\begin{gather*}
		\Delta B = |\frac{\partial B}{\partial m_0}|\Delta m_0 + |\frac{\partial B}{\partial m}|\Delta m + |\frac{\partial B}{\partial I}|\Delta I = \frac{g}{IL}(\Delta m + \Delta m_0 + \frac{m-m_0}{I}\Delta I)
\end{gather*}
gdzie $\Delta m_0 =\Delta m = 0,01$ g oraz $\Delta I = 0,02$ A. (to jeszcze do sprawdzenia czy takie niepewności były)


\end{document}