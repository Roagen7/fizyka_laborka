\documentclass{article}

\usepackage[T1]{fontenc}
\usepackage[polish]{babel}
\usepackage[utf8]{inputenc}
\usepackage{amsmath}
\usepackage{hyperref}
\usepackage{tablefootnote}
\begin{document}

\section{Cel ćwiczenia}
Celem ćwiczenia było wyznaczenie gęstości wybranych cieczy  przy użyciu wagi Mohra oraz piknometru dla denaturatu i zasolonej wody.  W obu pomiarach jako gęstość wzorcową przyjęto gęstość wody destylowanej $\rho = 1000 \frac{kg}{m^3}$.

\section{Opracowanie wyników dla wagi Mohra}
\subsection{pomierzone dane}
\begin{center}
\begin{tabular}{ c | c | c | c }
badana ciecz & p & q & r \\
\hline
 woda destylowana & 1; 8 & 9  &  9  \\ 
 denaturat  & 1; 7 & 3 & - \\  
 słona woda & 2; 9 & 7 & -    

\end{tabular}
\end{center}
$p$ - numer konika,  na którym zawieszono masę $m$ (jeśli użyto kilku mas, to numery podane po średniku)\\
$q$ - numer konika,  na którym zawieszono masę $\frac{m}{10}$ \\
$r$ - numer konika, na którym zawieszono masę $\frac{m}{100}$
\subsection{obliczenie gęstości}
korzystamy ze wzoru\footnote{\url{https://pg.edu.pl/files/ftims/2021-03/cwiczenieM1.pdf} (M1.4)}
\begin{gather*}
	\rho = \rho_w \frac{p+\frac{q}{10} + \frac{r}{100}}{p_w+\frac{q_w}{10} + \frac{r_w}{100}}
\end{gather*} 
\\ $\rho_w = 1000 \frac{kg}{m^3}$ (ciecz wzorcowa - woda destylowana)
\\ $p, q, r$ - pomierzone odległości koników od punktu podparcia dźwigni dla badanej cieczy \\
$p_w, q_w,  r_w$ - pomierzone odległości koników od punktu podparcia dźwigni dla cieczy wzorcowej (woda destylowana)

\begin{center}
\begin{tabular}{ c | c }
badana ciecz & pomierzona gęstość $\rho [\frac{kg}{m^3}]$\\
\hline
 denaturat  & 831,83\\  
 słona woda & 1173,17
\end{tabular}
\end{center}

\subsection{rachunek niepewności}
do wyliczenia niepewności korzystamy ze wzoru \footnote{\url{https://pg.edu.pl/files/ftims/2021-03/cw_26.pdf} (26.9)}
\begin{gather*}
	|\frac{\Delta \rho}{\rho}| = |\frac{\frac{\Delta r_w}{100}}{p_w+\frac{q_w}{10} + \frac{r_w}{100}}| + |\frac{\frac{\Delta q}{10}}{p+\frac{q}{10} + \frac{r}{100}}|
\end{gather*}
gdzie $\Delta r_w = \Delta q = 1/2$ (używamy $\Delta q$ - w pomiarach nie używaliśmy masy $\frac{m}{100}$)

\begin{center}
\begin{tabular}{ c | c }
badana ciecz & niepewność |$\frac{\Delta \rho}{\rho}$| [\%] \\
\hline
denaturat  & 0,7\\  
słona woda & 0,5
\end{tabular}
\end{center}

\section{Opracowanie wyników dla Piknometru}

\subsection{pomierzone dane}
\begin{center}
\begin{tabular}{ c | c }
pomiar & pomierzona masa [g]  \\
\hline
sam piknometr &  26,972  \\ 
piknometr z wodą destylowaną& 78,325  \\  
piknometr z denaturatem & 69,541 \\
piknometr z wodą morską & 86,460    
\end{tabular}
\end{center}

\subsection{obliczenie gęstości}

korzystamy ze wzoru \footnote{\url{https://pg.edu.pl/files/ftims/2021-03/cwiczenieM1.pdf} (M1.7)} 
\begin{gather*}
	\rho = \rho_w \frac{m_3 - m_1}{m_2 - m_1}
\end{gather*}
$m_1$ - masa samego piknometru \\
$m_2$ - masa piknometru z cieczą wzorcową (woda destylowana) \\
$m_3$ - masa piknometru z badaną cieczą \\
$\rho_w$ - gęstość wzorcowa (wody destylowanej, tak jak poprzednio)

\begin{center}
\begin{tabular}{ c | c }
badana ciecz & pomierzona gęstość $\rho [\frac{kg}{m^3}]$\\
\hline
 denaturat  & 828,95 \\  
 słona woda & 1158,41
\end{tabular}
\end{center}
\subsection{rachunek niepewności}
do wyliczenia niepewności korzystamy z oszcowania \footnote{\url{https://pg.edu.pl/files/ftims/2021-03/cw_26.pdf} (26.11)} \\
\begin{gather*}
	|\frac{\Delta\rho}{\rho}| \approx 3|\frac{\Delta m}{m_2 - m_1}|
\end{gather*}
gdzie $\Delta m = 0,001$
czyli |$\frac{\Delta\rho}{\rho}|  \approx 0,006\%$
\\ \\ \\
\section{Wnioski}

\begin{center}
\begin{tabular}{ c | c | c | c }
rodzaj cieczy & waga Mohra  $\rho [\frac{kg}{m^3}]$ & piknometr $\rho [\frac{kg}{m^3}]$ & tablice $\rho [\frac{kg}{m^3}] $ \\

\hline
 denaturat  & 831,83 $\pm$ 5,82 &  828,95 $\pm$  0,05& 800\\  
 słona woda & 1173,17 $\pm$ 5,87&1158,41 $\pm$ 0,07 & 1030
\end{tabular}
\end{center}

Mimo, że występuje odstępstwo od wskazań z tablic\footnote{\url{http://fizyka.edu.pl/gestosc-substancji/}}, wyniki pomiarów piknometrem i wagą Mohra są do siebie zbliżone.  Sugerowałoby to,  że stężenie soli w zasolonej wodzie różni się od tego w wodzie morskiej a denaturat mógł być trochę rozcieńczony wodą po wielokrotnym użytku (stąd w obu sytuacjach wzrost gęstości w stosunku do tej z tablic).  Różnice między wartościami spowodowane są niedokładnością techniki wykonywania pomiarów \\ np.  niewystarczające wysuszenie środka piknometru i nagrzanie piknometru suszarką co spowodowało podgrzanie cieczy w rezultacie zmniejszając wyznaczoną gęstość.  W przypadku wagi Mohra na dokładność mogła wpłynąć także trudność w określeniu położenia równowagi, oscylacje ramienia dźwigni oraz przyjęte założenie, że nurek jest nieważki.


\end{document}