\documentclass{article}

\usepackage[T1]{fontenc}
\usepackage[polish]{babel}
\usepackage[utf8]{inputenc}

\begin{document}

\section{Cel ćwiczenia}
Celem ćwiczenia jest wyznaczenie gęstości wybranych cieczy  przy użyciu wagi Mohra oraz piknometru.

\section{Opracowanie wyników dla wagi Mohra}
\subsection{pomierzone dane}
\begin{center}
\begin{tabular}{ c | c | c | c }
badana ciecz & p & q & r \\
 woda destylowana & 1; 8 & 9  &  9  \\ 
 denaturat  & 1; 7 & 3 & - \\  
 słona woda & 2; 9 & 7 & -    
\end{tabular}
\end{center}

\subsection{obliczenie gęstości}
korzystamy ze wzoru 
$\rho = \rho_w \frac{p+\frac{q}{10} + \frac{r}{100}}{p_w+\frac{q_w}{10} + \frac{r_w}{100}} $
\\ ciecz wzorcowa = woda destylowana ($\rho_w = 1000 \frac{kg}{m^3}$)
\\ $p, q, r$ - pomierzone odległości koników od punktu podparcia dźwigni dla badanej cieczy \\
$p_w, q_w,  r_w$ - pomierzone odległości koników od punktu podparcia dźwigni dla cieczy wzorcowej (woda destylowana)

\begin{center}
\begin{tabular}{ c | c }
badana ciecz & pomierzona gęstość $\rho [\frac{kg}{m^3}]$\\
 denaturat  & 831, 831\\  
 słona woda & 1173,173
\end{tabular}
\end{center}

\subsection{rachunek niepewności}
do wyliczenia niepewności korzystamy ze wzoru $|\frac{\Delta \rho}{\rho}| = |\frac{\frac{\Delta r_w}{100}}{p_w+\frac{q_w}{10} + \frac{r_w}{100}}| + |\frac{\frac{\Delta q}{10}}{p+\frac{q}{10} + \frac{r}{100}}|$ \\
$\Delta r_w = \Delta q = 1/2$

\begin{center}
\begin{tabular}{ c | c }
badana ciecz & niepewność |$\frac{\Delta \rho}{\rho}$| [\%] \\
denaturat  & 0,60241\\  
słona woda & 0,47774
\end{tabular}
\end{center}




\section{Opracowanie wyników dla Piknometru}

\subsection{pomierzone dane}
\begin{center}
\begin{tabular}{ c | c }
pomiar & pomierzona masa [g]  \\
sam piknometr &  26,972  \\ 
piknometr z wodą destylowaną& 78,325  \\  
piknometr z denaturatem & 69,541 \\
piknometr z wodą morską & 86,460    
\end{tabular}
\end{center}

\subsection{obliczenie gęstości}

korzystamy ze wzoru $\rho = \rho_w \frac{m_3 - m_1}{m_2 - m_1}$ \\
$m_1$ - masa samego piknometru \\
$m_2$ - masa piknometru z cieczą wzorcową (woda destylowana) \\
$m_3$ - masa piknometru z badaną cieczą \\
$\rho_w$ - gęstość wzorcowa (wody destylowanej, tak jak poprzednio)

\begin{center}
\begin{tabular}{ c | c }
badana ciecz & pomierzona gęstość $\rho [\frac{kg}{m^3}]$\\
 denaturat  & 828 9486 \\  
 słona woda & 1158,41334
\end{tabular}
\end{center}
\subsection{rachunek niepewności}
do wyliczenia niepewności korzystamy ze wzoru \\

\section{Wnioski}

\end{document}