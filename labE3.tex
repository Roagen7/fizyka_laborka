\documentclass{article}

\usepackage[T1]{fontenc}
\usepackage[polish]{babel}
\usepackage[utf8]{inputenc}
\usepackage{amsmath}
\usepackage{hyperref}
\usepackage{tablefootnote}
\usepackage{graphicx}
\usepackage{siunitx}
\graphicspath{ {./media/} }
\begin{document}

\section{Cel ćwiczenia}
Celem cwiczenia jest pomiar oporu elektrycznego pojedynczych rezystorów
oraz układu rezystorów połaczonych szeregowo i równolegle z wykorzystaniem
mostka pradu stałego (mostek Wheatstone’a).

\section{Badanie rezystancji pojedynczych rezystorów o nieznanej wartości}
\subsection{pomierzone dane}
\begin{center}
\begin{tabular}{ c | c | c | c}
Rezystor & $R_n$ [\si{\ohm}] & $l_1$ [mm] & $l_2$ [mm]\\
\hline
 1    & 152 & 480 & 520\\ 
 2    & 620 & 506 & 494\\ 
 3  & 430 & 504 & 496\\ 
 4  & 2040 & 501 & 499\\
 5   & 3030 & 502 & 498\\
 6  & 13800 & 500 & 500\\
 
\end{tabular}
\end{center}
$R_n$ - opór wzorcowy mostka\\ 
$l_1$ - położenie ślizgacza na skali milimetrowej listwy\\

\subsection{obliczenie rezystancji}
korzystamy ze wzoru\footnote{\url{https://pg.edu.pl/files/ftims/2021-03/cwiczenieE3.pdf} (E3.8)}
\begin{gather*}
	R_x = R_n\frac{l_1}{l_2}
\end{gather*} 
$R_x$ - opór badanego rezystora\\ 
$R_n$ - opór wzorcowy mostka\\ 
$l_1$ - położenie ślizgacza na skali milimetrowej listwy\\




\section{Badanie rezystancji układów rezystorów połączonych szeregowo}
\subsection{pomierzone dane}
\begin{center}
\begin{tabular}{ c |  c | c | c}
Rezystor & $R_n$ [\si{\ohm}] & $l_1$ [mm] & $l_2$ [mm]\\
\hline
 8 i 9 & 5000 & 508 & 492\\ 
 8 i 5 & 2720 & 500 & 500\\ 
 9 i 5 & 3780 & 500 & 500\\ 

\end{tabular}
\end{center}
$R_n$ - opór wzorcowy mostka\\ 
$l_1$ - położenie ślizgacza na skali milimetrowej listwy\\

\subsection{obliczenie rezystancji}
do obliczenia rezystancji korzystamy z tego samego wzoru co w pkt. 2.2\\
otrzymujemy: \\


\section{Badanie rezystancji układów rezystorów połączonych równolegle}
\subsection{pomierzone dane}
\begin{center}
\begin{tabular}{ c |  c | c | c}
Rezystor & $R_n$ [\si{\ohm}] & $l_1$ [mm] & $l_2$ [mm]\\
\hline
 8 i 9 & 1260 & 499 & 501\\ 
 8 i 5 & 495 & 500 & 500\\ 
 9 i 5 & 540 & 500 & 500\\ 

\end{tabular}
\end{center}
$R_n$ - opór wzorcowy mostka\\ 
$l_1$ - położenie ślizgacza na skali milimetrowej listwy\\

\section{Badanie drutów konstantanowych o różnej średnicy}
\subsection{pomierzone dane}

\begin{center}
\begin{tabular}{ c |  c | c | c}
d [mm] & $R_n$ [\si{\ohm}] & $l_1$ [mm] & $l_2$ [mm]\\
\hline
 0,35 & 5 & 503 & 497\\ 
 0,50 & 2 & 544 & 456\\ 
 0,70 & 1 & 547 & 453\\ 
 1,00 & 1 & 365 & 635\\ 

\end{tabular}
\end{center}
{d} - średnica drutu\\
$R_n$ - opór wzorcowy mostka\\ 
$l_1$ - położenie ślizgacza na skali milimetrowej listwy\\

\subsection {Zależność R =$ f(\frac{1}{d^2})$}

\subsection {obliczenie oporu właściwego konstantatu}

\section{Wnioski}
	



\end{document}