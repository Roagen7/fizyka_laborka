\documentclass{article}

\usepackage[T1]{fontenc}
\usepackage[polish]{babel}
\usepackage[utf8]{inputenc}
\usepackage{amsmath}
\usepackage{hyperref}
\usepackage{tablefootnote}
\begin{document}

\section{Cel ćwiczenia}
Celem ćwiczenia było wyznaczenie współczynnika sprężystości sprężyny przy użyciu metody statycznej i dynamicznej.  Ćwiczenie obejmowało również wyznaczenie współczynnika sprężystości dla układu sprężyn połączonych równolegle i szeregowo.

\section{Metoda statyczna}
\subsection{pomierzone dane}
\begin{center}
\begin{tabular}{ c | c | c }
l.p. & $\Delta x$ [cm] & m [g]\\
\hline
 1 & 3,7   & 50\\ 
 2 & 6,7   & 100\\ 
 3 & 10,3 & 150\\ 
 4 & 13, 9 & 200 \\
 5 & 17,0  & 250 \\
 6 & 20,6 & 300 \\
 7 & 24,7 & 350 \\
 8 & 27,7 & 400 \\
 9 & 31,3 & 450

\end{tabular}
\end{center}
$m$ - masa zawieszona na sprężynie \\
$\Delta x$ - zmiana wychylenia sprężyny od wychylenia początkowego po zawieszeniu ciężarków o określonej masie

\subsection{wykres $\Delta x(m)$}
miejsce na wykres

\subsection{obliczenie stałej sprężystości}
korzystamy ze wzoru\footnote{\url{https://pg.edu.pl/files/ftims/2021-03/cwiczenieM1.pdf} (M1.4)}
\begin{gather*}
	\rho = \rho_w \frac{p+\frac{q}{10} + \frac{r}{100}}{p_w+\frac{q_w}{10} + \frac{r_w}{100}}
\end{gather*} 

\subsection{rachunek niepewności}
do wyliczenia niepewności korzystamy ze wzoru \footnote{\url{https://pg.edu.pl/files/ftims/2021-03/cw_26.pdf} (26.9)}

\section{Metoda dynamiczna}
\subsection{pomierzone dane}
\begin{center}
\begin{tabular}{ c | c | c| c }
l.p. & $t$ [s] & m [g]  & $T^2$ [$s^2$]\\
\hline
 1 & 8,07  & 50\\ 
 2 & 10,64  & 100\\ 
 3 & 12,55  & 150\\ 
 4 & 15,19 & 200 \\
 5 & 16,66 & 250 \\
 6 & 17,94 & 300 \\
 7 & 19,85 & 350 \\
 8 & 21,41 & 400 \\

\end{tabular}
\end{center}
$m$ - masa zawieszona na sprężynie \\
$t$ - pomierzony czas 20 okresów \\
$t = 20T$

\subsection{wykres $T^2(m)$}
miejsce na wykres

\subsection{obliczenie stałej sprężystości}
korzystamy ze wzoru\footnote{\url{https://pg.edu.pl/files/ftims/2021-03/cwiczenieM1.pdf} (M1.4)}
\begin{gather*}
	\rho = \rho_w \frac{p+\frac{q}{10} + \frac{r}{100}}{p_w+\frac{q_w}{10} + \frac{r_w}{100}}
\end{gather*} 

\subsection{rachunek niepewności}
do wyliczenia niepewności korzystamy ze wzoru \footnote{\url{https://pg.edu.pl/files/ftims/2021-03/cw_26.pdf} (26.9)}

\section{Moduł sztywności}
\subsection{pomierzone dane}
\begin{center}
\begin{tabular}{ c | c  }
dana & wartość\\
\hline
 r & 0,35mm \\
 R & 7,05mm \\ 
 N & 80 zwojów \\ 

\end{tabular}
\end{center}
$r$ - promień drutu sprężyny \\
$R$ - promień sprężyny \\
$N$ - liczba zwojów sprężyny

\subsection{obliczenie modułu sztywności}
korzystamy ze wzoru\footnote{\url{https://pg.edu.pl/files/ftims/2021-03/cwiczenieM1.pdf} (M1.4)}
\begin{gather*}
	\rho = \rho_w \frac{p+\frac{q}{10} + \frac{r}{100}}{p_w+\frac{q_w}{10} + \frac{r_w}{100}}
\end{gather*} 

\subsection{rachunek niepewności}
do wyliczenia niepewności korzystamy ze wzoru \footnote{\url{https://pg.edu.pl/files/ftims/2021-03/cw_26.pdf} (26.9)}

\section{Układ sprężyn połączony równolegle}
\subsection{pomierzone dane}
\begin{center}
\begin{tabular}{ c | c | c | c | c | c }
l.p. & $\Delta x$ [cm] & m [g] & t[s]  & T[s] & $T^2$ [$s^2$]\\
\hline
 1 & 2,2   & 50 & 8,06\\ 
 2 & 4,3   & 100 & 9,41\\ 
 3 & 6,6   & 150 & 11,45\\ 
 4 & 8,8  & 200 & 13,02\\
 5 & 10,9 & 250 & 14,11\\
 6 & 12,7 & 300 & 15,53\\
 7 & 14,6 & 350 & 16,46\\
 8 & 17,1  & 400 & 17,62\\
 9 & 19,2 & 450 & 18,39

\end{tabular}
\end{center}
oznaczenia jak w pozostałych podpunktach

\subsection{wykres i opracowanie $\Delta x(m)$}
miejsce na wykres

\subsection{wykres i opracowanie $T^2(m)$}
miejsce na wykres

\subsection{rachunek niepewności}
do wyliczenia niepewności korzystamy ze wzoru \footnote{\url{https://pg.edu.pl/files/ftims/2021-03/cw_26.pdf} (26.9)}

\section{Układ sprężyn połączony szeregowo}
\subsection{pomierzone dane}
\begin{center}
\begin{tabular}{ c | c | c | c | c | c }
l.p. & $\Delta x$ [cm] & m [g] & t[s]  & T[s] & $T^2$ [$s^2$]\\
\hline
 1 & 8,8   & 50 & 13,42 \\ 
 2 & 18,0   & 100 &17,80\\ 
 3 & 27,1   & 150 & 21,13\\ 
 4 & 35,9  & 200 & 24,49\\
 5 & 44,9 & 250 & 27,17\\


\end{tabular}
\end{center}
oznaczenia jak w pozostałych podpunktach

\subsection{wykres i opracowanie $\Delta x(m)$}
miejsce na wykres

\subsection{wykres i opracowanie $T^2(m)$}
miejsce na wykres

\subsection{rachunek niepewności}
do wyliczenia niepewności korzystamy ze wzoru \footnote{\url{https://pg.edu.pl/files/ftims/2021-03/cw_26.pdf} (26.9)}


\section{Wnioski}

\begin{center}
\begin{tabular}{ c | c }
zadanie & współczynnik spreżystości  \\

\hline
 metoda statyczna  & 831,83\\  
 metoda dynamiczna & 1173,17 \\
 układ połączony szeregowo & 123 \\
 układ połączony równolegle & 123 
\end{tabular}
\end{center}

Podobieństwo pomiaru metodą statyczną i dynamiczną.
Sensowność wyników w układzie połączonym równolegle i szeregowo (równolegle > szeregowo).
Moduł sztywności.
Błąd paralaksy, błąd reakcji, niedokładność linijki.


\end{document}