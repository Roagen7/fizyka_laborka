\documentclass{article}

\usepackage[T1]{fontenc}
\usepackage[polish]{babel}
\usepackage[utf8]{inputenc}
\usepackage{amsmath}
\usepackage{hyperref}
\usepackage{tablefootnote}
\begin{document}

\section{Cel ćwiczenia}
Celem ćwiczenia było wyznaczenie współczynnika sprężystości sprężyny przy użyciu metody statycznej i dynamicznej.  Ćwiczenie obejmowało również wyznaczenie współczynnika sprężystości dla układu sprężyn połączonych równolegle i szeregowo.

\section{Opracowanie wyników}
\subsection{pomierzone dane}
\begin{center}
\begin{tabular}{ c | c | c | c }
badana ciecz & p & q & r \\
\hline
 woda destylowana & 1; 8 & 9  &  9  \\ 
 denaturat  & 1; 7 & 3 & - \\  
 słona woda & 2; 9 & 7 & -    

\end{tabular}
\end{center}
$p$ - numer konika,  na którym zawieszono masę $m$ (jeśli użyto kilku mas, to numery podane po średniku)\\
$q$ - numer konika,  na którym zawieszono masę $\frac{m}{10}$ \\
$r$ - numer konika, na którym zawieszono masę $\frac{m}{100}$
\subsection{obliczenie gęstości}
korzystamy ze wzoru\footnote{\url{https://pg.edu.pl/files/ftims/2021-03/cwiczenieM1.pdf} (M1.4)}
\begin{gather*}
	\rho = \rho_w \frac{p+\frac{q}{10} + \frac{r}{100}}{p_w+\frac{q_w}{10} + \frac{r_w}{100}}
\end{gather*} 

\subsection{rachunek niepewności}
do wyliczenia niepewności korzystamy ze wzoru \footnote{\url{https://pg.edu.pl/files/ftims/2021-03/cw_26.pdf} (26.9)}

\section{Wnioski}

\begin{center}
\begin{tabular}{ c | c | c | c }
rodzaj cieczy & waga Mohra  $\rho [\frac{kg}{m^3}]$ & piknometr $\rho [\frac{kg}{m^3}]$ & tablice $\rho [\frac{kg}{m^3}] $ \\

\hline
 denaturat  & 831,83 $\pm$ 5,82 &  828,95 $\pm$  0,05& 800\\  
 słona woda & 1173,17 $\pm$ 5,87&1158,41 $\pm$ 0,07 & 1030
\end{tabular}
\end{center}

Mimo, że występuje odstępstwo od wskazań z tablic\footnote{\url{http://fizyka.edu.pl/gestosc-substancji/}}, wyniki pomiarów piknometrem i wagą Mohra są do siebie zbliżone.  Sugerowałoby to,  że stężenie soli w zasolonej wodzie różni się od tego w wodzie morskiej a denaturat mógł być trochę rozcieńczony wodą po wielokrotnym użytku (stąd w obu sytuacjach wzrost gęstości w stosunku do tej z tablic).  Różnice między wartościami spowodowane są niedokładnością techniki wykonywania pomiarów \\ np.  niewystarczające wysuszenie środka piknometru i nagrzanie piknometru suszarką co spowodowało podgrzanie cieczy w rezultacie zmniejszając wyznaczoną gęstość.  W przypadku wagi Mohra na dokładność mogła wpłynąć także trudność w określeniu położenia równowagi, oscylacje ramienia dźwigni oraz przyjęte założenie, że nurek jest nieważki.


\end{document}